% Options for packages loaded elsewhere
\PassOptionsToPackage{unicode}{hyperref}
\PassOptionsToPackage{hyphens}{url}
%
\documentclass[
]{article}
\usepackage{amsmath,amssymb}
\usepackage{iftex}
\ifPDFTeX
  \usepackage[T1]{fontenc}
  \usepackage[utf8]{inputenc}
  \usepackage{textcomp} % provide euro and other symbols
\else % if luatex or xetex
  \usepackage{unicode-math} % this also loads fontspec
  \defaultfontfeatures{Scale=MatchLowercase}
  \defaultfontfeatures[\rmfamily]{Ligatures=TeX,Scale=1}
\fi
\usepackage{lmodern}
\ifPDFTeX\else
  % xetex/luatex font selection
\fi
% Use upquote if available, for straight quotes in verbatim environments
\IfFileExists{upquote.sty}{\usepackage{upquote}}{}
\IfFileExists{microtype.sty}{% use microtype if available
  \usepackage[]{microtype}
  \UseMicrotypeSet[protrusion]{basicmath} % disable protrusion for tt fonts
}{}
\makeatletter
\@ifundefined{KOMAClassName}{% if non-KOMA class
  \IfFileExists{parskip.sty}{%
    \usepackage{parskip}
  }{% else
    \setlength{\parindent}{0pt}
    \setlength{\parskip}{6pt plus 2pt minus 1pt}}
}{% if KOMA class
  \KOMAoptions{parskip=half}}
\makeatother
\usepackage{xcolor}
\usepackage[margin=1in]{geometry}
\usepackage{color}
\usepackage{fancyvrb}
\newcommand{\VerbBar}{|}
\newcommand{\VERB}{\Verb[commandchars=\\\{\}]}
\DefineVerbatimEnvironment{Highlighting}{Verbatim}{commandchars=\\\{\}}
% Add ',fontsize=\small' for more characters per line
\usepackage{framed}
\definecolor{shadecolor}{RGB}{248,248,248}
\newenvironment{Shaded}{\begin{snugshade}}{\end{snugshade}}
\newcommand{\AlertTok}[1]{\textcolor[rgb]{0.94,0.16,0.16}{#1}}
\newcommand{\AnnotationTok}[1]{\textcolor[rgb]{0.56,0.35,0.01}{\textbf{\textit{#1}}}}
\newcommand{\AttributeTok}[1]{\textcolor[rgb]{0.13,0.29,0.53}{#1}}
\newcommand{\BaseNTok}[1]{\textcolor[rgb]{0.00,0.00,0.81}{#1}}
\newcommand{\BuiltInTok}[1]{#1}
\newcommand{\CharTok}[1]{\textcolor[rgb]{0.31,0.60,0.02}{#1}}
\newcommand{\CommentTok}[1]{\textcolor[rgb]{0.56,0.35,0.01}{\textit{#1}}}
\newcommand{\CommentVarTok}[1]{\textcolor[rgb]{0.56,0.35,0.01}{\textbf{\textit{#1}}}}
\newcommand{\ConstantTok}[1]{\textcolor[rgb]{0.56,0.35,0.01}{#1}}
\newcommand{\ControlFlowTok}[1]{\textcolor[rgb]{0.13,0.29,0.53}{\textbf{#1}}}
\newcommand{\DataTypeTok}[1]{\textcolor[rgb]{0.13,0.29,0.53}{#1}}
\newcommand{\DecValTok}[1]{\textcolor[rgb]{0.00,0.00,0.81}{#1}}
\newcommand{\DocumentationTok}[1]{\textcolor[rgb]{0.56,0.35,0.01}{\textbf{\textit{#1}}}}
\newcommand{\ErrorTok}[1]{\textcolor[rgb]{0.64,0.00,0.00}{\textbf{#1}}}
\newcommand{\ExtensionTok}[1]{#1}
\newcommand{\FloatTok}[1]{\textcolor[rgb]{0.00,0.00,0.81}{#1}}
\newcommand{\FunctionTok}[1]{\textcolor[rgb]{0.13,0.29,0.53}{\textbf{#1}}}
\newcommand{\ImportTok}[1]{#1}
\newcommand{\InformationTok}[1]{\textcolor[rgb]{0.56,0.35,0.01}{\textbf{\textit{#1}}}}
\newcommand{\KeywordTok}[1]{\textcolor[rgb]{0.13,0.29,0.53}{\textbf{#1}}}
\newcommand{\NormalTok}[1]{#1}
\newcommand{\OperatorTok}[1]{\textcolor[rgb]{0.81,0.36,0.00}{\textbf{#1}}}
\newcommand{\OtherTok}[1]{\textcolor[rgb]{0.56,0.35,0.01}{#1}}
\newcommand{\PreprocessorTok}[1]{\textcolor[rgb]{0.56,0.35,0.01}{\textit{#1}}}
\newcommand{\RegionMarkerTok}[1]{#1}
\newcommand{\SpecialCharTok}[1]{\textcolor[rgb]{0.81,0.36,0.00}{\textbf{#1}}}
\newcommand{\SpecialStringTok}[1]{\textcolor[rgb]{0.31,0.60,0.02}{#1}}
\newcommand{\StringTok}[1]{\textcolor[rgb]{0.31,0.60,0.02}{#1}}
\newcommand{\VariableTok}[1]{\textcolor[rgb]{0.00,0.00,0.00}{#1}}
\newcommand{\VerbatimStringTok}[1]{\textcolor[rgb]{0.31,0.60,0.02}{#1}}
\newcommand{\WarningTok}[1]{\textcolor[rgb]{0.56,0.35,0.01}{\textbf{\textit{#1}}}}
\usepackage{graphicx}
\makeatletter
\def\maxwidth{\ifdim\Gin@nat@width>\linewidth\linewidth\else\Gin@nat@width\fi}
\def\maxheight{\ifdim\Gin@nat@height>\textheight\textheight\else\Gin@nat@height\fi}
\makeatother
% Scale images if necessary, so that they will not overflow the page
% margins by default, and it is still possible to overwrite the defaults
% using explicit options in \includegraphics[width, height, ...]{}
\setkeys{Gin}{width=\maxwidth,height=\maxheight,keepaspectratio}
% Set default figure placement to htbp
\makeatletter
\def\fps@figure{htbp}
\makeatother
\setlength{\emergencystretch}{3em} % prevent overfull lines
\providecommand{\tightlist}{%
  \setlength{\itemsep}{0pt}\setlength{\parskip}{0pt}}
\setcounter{secnumdepth}{-\maxdimen} % remove section numbering
\ifLuaTeX
  \usepackage{selnolig}  % disable illegal ligatures
\fi
\IfFileExists{bookmark.sty}{\usepackage{bookmark}}{\usepackage{hyperref}}
\IfFileExists{xurl.sty}{\usepackage{xurl}}{} % add URL line breaks if available
\urlstyle{same}
\hypersetup{
  pdftitle={Discrete Random Variables},
  hidelinks,
  pdfcreator={LaTeX via pandoc}}

\title{Discrete Random Variables}
\author{}
\date{\vspace{-2.5em}}

\begin{document}
\maketitle

\begin{Shaded}
\begin{Highlighting}[]
\FunctionTok{library}\NormalTok{(ggplot2)}
\FunctionTok{library}\NormalTok{(dplyr)}
\end{Highlighting}
\end{Shaded}

\begin{verbatim}
## 
## Attaching package: 'dplyr'
\end{verbatim}

\begin{verbatim}
## The following objects are masked from 'package:stats':
## 
##     filter, lag
\end{verbatim}

\begin{verbatim}
## The following objects are masked from 'package:base':
## 
##     intersect, setdiff, setequal, union
\end{verbatim}

\begin{center}\rule{0.5\linewidth}{0.5pt}\end{center}

\textbf{Scenario}: A salesman has scheduled two appointments to sell
software, one in the morning and another one in the afternoon. There are
two software editions available: the base edition costing Rs. 500 and
the premium edition costing Rs. 1000. His morning appointments typically
lead to a sale with a 30\% chance while the afternoon ones typically
lead to a sale with a 60\% chance independent of what happened in the
morning. If the morning appointment ends up in sale, the salesman has a
70\% chance of selling the premium edition and if the afternoon
appointment ends up in a sale, he is equally likely to sell either of
the editions. Let X be the random variable representing the total Rupee
value of sales. What are the different values that X can take? Calculate
the probability that X takes the value 5000?

\begin{center}\rule{0.5\linewidth}{0.5pt}\end{center}

\begin{Shaded}
\begin{Highlighting}[]
\CommentTok{\# Sampling space for appointment success (0 corresponds to no sale, 1 corresponds to a sale)}
\NormalTok{s\_appointment }\OtherTok{=} \FunctionTok{c}\NormalTok{(}\DecValTok{0}\NormalTok{,}\DecValTok{1}\NormalTok{)}

\CommentTok{\# Appointment success and failure probabilities}
\NormalTok{p\_morning }\OtherTok{=} \FloatTok{0.30} \CommentTok{\# Success probability of sales in morning}
\NormalTok{p\_afternoon }\OtherTok{=} \FloatTok{0.60}  \CommentTok{\# Success probability of sales in afternoon}
\NormalTok{p\_appointment }\OtherTok{=} \FunctionTok{matrix}\NormalTok{(}\AttributeTok{nrow  =} \DecValTok{2}\NormalTok{, }\AttributeTok{ncol =} \DecValTok{2}\NormalTok{, }\FunctionTok{c}\NormalTok{(}\DecValTok{1}\SpecialCharTok{{-}}\NormalTok{p\_morning,p\_morning ,}\DecValTok{1}\SpecialCharTok{{-}}\NormalTok{p\_afternoon ,p\_afternoon), }\AttributeTok{byrow =} \ConstantTok{TRUE}\NormalTok{)}

\CommentTok{\# Sampling space for software type}
\NormalTok{s\_software }\OtherTok{=} \FunctionTok{c}\NormalTok{(}\DecValTok{5000}\NormalTok{,}\DecValTok{10000}\NormalTok{)}
  
\CommentTok{\# Software type probabilities}
\NormalTok{p\_software}\OtherTok{=}\FunctionTok{matrix}\NormalTok{(}\AttributeTok{nrow=}\DecValTok{2}\NormalTok{,}\AttributeTok{ncol=}\DecValTok{2}\NormalTok{,}\FunctionTok{c}\NormalTok{(}\FloatTok{0.3}\NormalTok{,}\FloatTok{0.7}\NormalTok{,}\FloatTok{0.5}\NormalTok{,}\FloatTok{0.5}\NormalTok{),}\AttributeTok{byrow =} \ConstantTok{TRUE}\NormalTok{)}

\CommentTok{\# Function that simulates one trial of the random experiment which is}
\CommentTok{\# what the salesman earns on a random day}
\NormalTok{salesResult }\OtherTok{=} \ControlFlowTok{function}\NormalTok{()\{}
\NormalTok{  result }\OtherTok{=} \FunctionTok{numeric}\NormalTok{(}\DecValTok{2}\NormalTok{)}
  \CommentTok{\# Simulate whether sales happen in morning and afternoon appointments}
  \ControlFlowTok{for}\NormalTok{ (j }\ControlFlowTok{in} \FunctionTok{c}\NormalTok{(}\DecValTok{1}\SpecialCharTok{:}\DecValTok{2}\NormalTok{))\{}
\NormalTok{    result[j] }\OtherTok{=} \FunctionTok{sample}\NormalTok{(s\_appointment,}\AttributeTok{size=}\DecValTok{1}\NormalTok{,}\AttributeTok{replace=}\ConstantTok{TRUE}\NormalTok{,}\AttributeTok{prob=}\NormalTok{p\_appointment[j, ])}
\NormalTok{  \}}
\NormalTok{  earnings }\OtherTok{=} \FunctionTok{ifelse}\NormalTok{(result[}\DecValTok{1}\NormalTok{]}\SpecialCharTok{==}\DecValTok{1}\NormalTok{,}\FunctionTok{sample}\NormalTok{(s\_software,}\AttributeTok{size=}\DecValTok{1}\NormalTok{, }\AttributeTok{replace=}\ConstantTok{TRUE}\NormalTok{,}\AttributeTok{prob=}\NormalTok{p\_software[}\DecValTok{1}\NormalTok{,]),}\DecValTok{0}\NormalTok{)}\SpecialCharTok{+}\FunctionTok{ifelse}\NormalTok{(result[}\DecValTok{2}\NormalTok{]}\SpecialCharTok{==}\DecValTok{1}\NormalTok{,}\FunctionTok{sample}\NormalTok{(s\_software,}\AttributeTok{size=}\DecValTok{1}\NormalTok{,}\AttributeTok{replace=}\ConstantTok{TRUE}\NormalTok{,}\AttributeTok{prob=}\NormalTok{p\_software[}\DecValTok{2}\NormalTok{, ]),}\DecValTok{0}\NormalTok{)}
  \FunctionTok{return}\NormalTok{(earnings)}
\NormalTok{\}}

\CommentTok{\# Number of simulations}
\NormalTok{nsimulations }\OtherTok{=} \FloatTok{1e5}
\NormalTok{simulatedData }\OtherTok{=} \FunctionTok{replicate}\NormalTok{(nsimulations, }\FunctionTok{salesResult}\NormalTok{())}

\CommentTok{\# probability that the salesman takes 5000}
\FunctionTok{mean}\NormalTok{(simulatedData}\SpecialCharTok{==}\DecValTok{5000}\NormalTok{)}
\end{Highlighting}
\end{Shaded}

\begin{verbatim}
## [1] 0.245
\end{verbatim}

\begin{Shaded}
\begin{Highlighting}[]
\CommentTok{\# probability that  the salesman earns}
\FunctionTok{table}\NormalTok{(simulatedData)}\SpecialCharTok{/}\NormalTok{nsimulations}
\end{Highlighting}
\end{Shaded}

\begin{verbatim}
## simulatedData
##       0    5000   10000   15000   20000 
## 0.28225 0.24500 0.31939 0.08951 0.06385
\end{verbatim}

\begin{center}\rule{0.5\linewidth}{0.5pt}\end{center}

\textbf{Discrete random variable}: Let \(X\) represent thetotal earnings
from one day whic is random; that is, \(X\) is a discrete random
variable which can take the values \(0, 5000, 10000, 15000, 20000.\) The
associated probabilities can be calculated as:

\(\begin{align*}P(X = 0) &= 0.28,\\ P(X = 5000) &= 0.31,\\ P(X = 10000) &= 0.275,\\ P(X = 15000) &= 0.095,\\ P(X = 20000) &= 0.04.\end{align*}\)

\begin{center}\rule{0.5\linewidth}{0.5pt}\end{center}

\textbf{Probability Mass Function (PMF)} of the random variable \(X\) is
denoted as \(P_X(x),\) where \(x\) represents the possible values that
the random variable \(X\) can take:

\(\begin{align*}{\color{red}{P_X}}(0) &= P(X = 0) = 0.28,\\ {\color{red}{P_X}}(5000) &=P(X = 5000) = 0.31,\\ {\color{red}{P_X}}(10000) &=P(X = 10000) = 0.275,\\ {\color{red}{P_X}}(15000) &=P(X = 15000) = 0.095,\\ {\color{red}{P_X}}(20000) &=P(X = 20000) = 0.04.\end{align*}\)

\begin{center}\rule{0.5\linewidth}{0.5pt}\end{center}

\begin{Shaded}
\begin{Highlighting}[]
\CommentTok{\# Convert simulatedData into a dataframe}
\NormalTok{dfSales }\OtherTok{=} \FunctionTok{as.data.frame}\NormalTok{(simulatedData)}


\FunctionTok{colnames}\NormalTok{(dfSales) }\OtherTok{=}\StringTok{\textquotesingle{}Earning\textquotesingle{}}


\NormalTok{p }\OtherTok{=} \FunctionTok{ggplot}\NormalTok{(}\AttributeTok{data=}\NormalTok{dfSales)}\SpecialCharTok{+}\FunctionTok{geom\_bar}\NormalTok{(}\FunctionTok{aes}\NormalTok{(}\AttributeTok{x=}\NormalTok{Earning),}\AttributeTok{fill=}\StringTok{\textquotesingle{}steelblue\textquotesingle{}}\NormalTok{)}
\NormalTok{p}
\end{Highlighting}
\end{Shaded}

\includegraphics{DiscreteRandomVariables_files/figure-latex/unnamed-chunk-3-1.pdf}

\begin{Shaded}
\begin{Highlighting}[]
\NormalTok{dfSales }\OtherTok{=} \FunctionTok{as.data.frame}\NormalTok{(}\FunctionTok{table}\NormalTok{(simulatedData))}
\CommentTok{\# Add names to the columns}
\FunctionTok{colnames}\NormalTok{(dfSales) }\OtherTok{=}\FunctionTok{c}\NormalTok{(}\StringTok{\textquotesingle{}Earning\textquotesingle{}}\NormalTok{,}\StringTok{\textquotesingle{}Frequency\textquotesingle{}}\NormalTok{)}
\NormalTok{dfSales[}\StringTok{\textquotesingle{}Probability\textquotesingle{}}\NormalTok{] }\OtherTok{=}\NormalTok{ dfSales[}\StringTok{\textquotesingle{}Frequency\textquotesingle{}}\NormalTok{]}\SpecialCharTok{/}\NormalTok{nsimulations}
\FunctionTok{head}\NormalTok{(dfSales, }\DecValTok{5}\NormalTok{)}
\end{Highlighting}
\end{Shaded}

\begin{verbatim}
##   Earning Frequency Probability
## 1       0     28225     0.28225
## 2    5000     24500     0.24500
## 3   10000     31939     0.31939
## 4   15000      8951     0.08951
## 5   20000      6385     0.06385
\end{verbatim}

\begin{Shaded}
\begin{Highlighting}[]
\NormalTok{p }\OtherTok{=} \FunctionTok{ggplot}\NormalTok{(}\AttributeTok{data=}\NormalTok{dfSales)}\SpecialCharTok{+}\FunctionTok{geom\_col}\NormalTok{(}\FunctionTok{aes}\NormalTok{(}\AttributeTok{x=}\NormalTok{Earning,}\AttributeTok{y=}\NormalTok{Frequency),}\AttributeTok{fill=}\StringTok{\textquotesingle{}steelblue\textquotesingle{}}\NormalTok{)}
\NormalTok{p}
\end{Highlighting}
\end{Shaded}

\includegraphics{DiscreteRandomVariables_files/figure-latex/unnamed-chunk-4-1.pdf}

What is the salesman's expected earnings? What does it even mean to say
``expected earning?'' It is a single number denoted as \(E[X]\) and
referred to as the \emph{expected value of} \(X\) which can be
calculated as follows:

\begin{center}\rule{0.5\linewidth}{0.5pt}\end{center}

\begin{itemize}
\tightlist
\item
  \emph{From a simulation perspective}, it is simply an average of all
  the simulated earnings:
  \(\begin{align*}E[X] &\approx 0 \times\underbrace{\frac{\text{No. of times 0 appeared}}{\text{nsimulations}}}_{\text{approximation to }P_X(0)}+500 \times\underbrace{\frac{\text{No. of times 500 appeared}}{\text{nsimulations}}}_{\text{approximation to }P_X(500)}+1000 \times\underbrace{\frac{\text{No. of times 1000 appeared}}{\text{nsimulations}}}_{\text{approximation to }P_X(1000)}+1500 \times\underbrace{\frac{\text{No. of times 1500 appeared}}{\text{nsimulations}}}_{\text{approximation to }P_X(1500)}+2000 \times\underbrace{\frac{\text{No. of times 2000 appeared}}{\text{nsimulations}}}_{\text{approximation to }P_X(2000)}.\end{align*}\)
\end{itemize}

\begin{center}\rule{0.5\linewidth}{0.5pt}\end{center}

\begin{itemize}
\tightlist
\item
  \emph{From a theoretical perspective}, s it is defined as the weighted
  sum of the possible values \(X\) can take with the corresponding
  probabilities:
\end{itemize}

\(\begin{align*} E[X] &= \sum_xxP_X(x),\ \text{where }x = 0, 500, 1000, 1500, 2000\\\Rightarrow E[X] &= 0\times P_X(0)+500\times P_X(500)+1000\times P_X(1000)+1500\times P_X(1500)+2000\times P_X(2000).\end{align*}\)

\begin{center}\rule{0.5\linewidth}{0.5pt}\end{center}

So, we see that the expected value of \(X\) is approximately the
long-term average of the simulated (or realized) values of \(X.\)

\begin{Shaded}
\begin{Highlighting}[]
\CommentTok{\# Expected earnings using simulated values}
\FunctionTok{mean}\NormalTok{(simulatedData) }\CommentTok{\# long term average of the realizations of X}
\end{Highlighting}
\end{Shaded}

\begin{verbatim}
## [1] 7038.55
\end{verbatim}

\begin{Shaded}
\begin{Highlighting}[]
\CommentTok{\# Expected earnings using the theoretical definition}
\NormalTok{x }\OtherTok{=} \FunctionTok{c}\NormalTok{(}\DecValTok{0}\NormalTok{, }\DecValTok{500}\NormalTok{, }\DecValTok{1000}\NormalTok{, }\DecValTok{1500}\NormalTok{, }\DecValTok{2000}\NormalTok{)}
\NormalTok{p }\OtherTok{=} \FunctionTok{c}\NormalTok{(}\FloatTok{0.28}\NormalTok{, }\FloatTok{0.27}\NormalTok{, }\FloatTok{0.315}\NormalTok{, }\FloatTok{0.09}\NormalTok{, }\FloatTok{0.045}\NormalTok{)}
\FunctionTok{sum}\NormalTok{(x }\SpecialCharTok{*}\NormalTok{ p)}
\end{Highlighting}
\end{Shaded}

\begin{verbatim}
## [1] 675
\end{verbatim}

How much could the salesman's earnings vary from the expected value of
the earning calculated in the previous cell?

\begin{center}\rule{0.5\linewidth}{0.5pt}\end{center}

To answer this question, let's first look at the quantity
\(X-E[X] = X-675.\) Note that:

\begin{itemize}
\item
  this is also a random variable with the possible values
  \(0-675, 500-675, 1000-675, 1500-675, 2000-675\);
\item
  it can be thought of as the random variable that captures the
  deviation of \(X\) from its expected value;
\item
  the associated probabilities are still the same as:

  \begin{array}{c|c|c}
  \hline
  \color{blue}X&\color{blue}{X}-\color{cyan}{E[X]} = \color{blue}{X}-\color{cyan}{675}&\color{magenta}{\text{Probability}}\\
  \hline
  0 & -675 & 0.28\\
  500 & -175 & 0.27\\
  1000 & 325 & 0.315\\
  1500 & 825 & 0.09\\
  2000 & 1325 & 0.045
  \end{array}

  \begin{center}\rule{0.5\linewidth}{0.5pt}\end{center}

  Now, consider the quanityt \((X-E[X])^2 = (X-675)^2.\) Note that:
\item
  this is also a random variable with the possible values
  \((0-675)^2, (500-675)^2, (1000-675)^2, (1500-675)^2, (2000-675)^2\);
\item
  it can be thought of as the random variable that captures the squared
  deviation of \(X\) from its expected value;
\item
  the associated probabilities are still the same as:

  \begin{array}{c|c|c}
  \hline
  \color{blue}X&(\color{blue}{X}-\color{cyan}{E[X]})^2 = (\color{blue}{X}-\color{cyan}{675})^2&\color{magenta}{\text{Probability}}\\
  \hline
  0 & (-675)^2 & 0.28\\
  500 & (-175)^2 & 0.27\\
  1000 & (325)^2 & 0.315\\
  1500 & (825)^2& 0.09\\
  2000 & (1325)^2 & 0.045
  \end{array}
\end{itemize}

\begin{center}\rule{0.5\linewidth}{0.5pt}\end{center}

\(\color{green}{Variance}\) of the random variable \(X,\) denoted as
\(\text{var}[X],\) is the expected squared deviation of \(X\) from its
expected value. In simple terms, variance of \(X\) is the expected value
of the random variable \((X-E[X])^2.\)

\(\begin{align*}\text{Var}[X] &= E[\underbrace{(X-E[X])^2}_{\text{squared deviation random variable}}]\\&= \sum_x (x-E[X])^2\times P_X(x)\\& = (0-675)^2\times P_X(0)+(500-675)^2\times P_X(500)+(1000-675)^2\times P_X(1000)+(1500-675)^2\times P_X(1500)+(2000-675)^2\times P_X(2000)\\&=(0-675)^2\times 0.28+(500-675)^2\times 0.27+(1000-675)^2\times 0.315+(1500-675)^2\times 0.09+(2000-675)^2\times 0.045.\end{align*}\)

\begin{center}\rule{0.5\linewidth}{0.5pt}\end{center}

\(\color{green}{Standard\ deviation}\) of the random variable \(X,\)
denoted as \(\text{SD}[X],\) is defined as the square root of its
variance:

\(\begin{align*}\text{SD}[X] &= \sqrt{\text{Var}[X]} = \sqrt{E\left[(X-E[X])^2\right]}\\&=\sqrt{(0-675)^2\times 0.28+(500-675)^2\times 0.27+(1000-675)^2\times 0.315+(1500-675)^2\times 0.09+(2000-675)^2\times 0.045}.\end{align*}\)

\begin{Shaded}
\begin{Highlighting}[]
\CommentTok{\# Variance of the earnings using simulated values}
\CommentTok{\# long term average of this random variable (X{-}E[X])\^{}2}
\CommentTok{\#simulatedData {-} mean(simulatedData) \# simulated deviations}
\CommentTok{\#(simulatedData {-} mean(simulatedData))\^{}2 \# simulated squared deviations}
\FunctionTok{mean}\NormalTok{(simulatedData)}
\end{Highlighting}
\end{Shaded}

\begin{verbatim}
## [1] 7038.55
\end{verbatim}

\begin{Shaded}
\begin{Highlighting}[]
\FunctionTok{mean}\NormalTok{((simulatedData }\SpecialCharTok{{-}} \FunctionTok{mean}\NormalTok{(simulatedData))}\SpecialCharTok{\^{}}\DecValTok{2}\NormalTok{) }\CommentTok{\# variance calculated using simulation}
\end{Highlighting}
\end{Shaded}

\begin{verbatim}
## [1] 34202564
\end{verbatim}

\begin{Shaded}
\begin{Highlighting}[]
\CommentTok{\# Standard deviation of the earnings using simulated values}
\FunctionTok{sqrt}\NormalTok{(}\FunctionTok{mean}\NormalTok{((simulatedData }\SpecialCharTok{{-}} \FunctionTok{mean}\NormalTok{(simulatedData))}\SpecialCharTok{\^{}}\DecValTok{2}\NormalTok{))}
\end{Highlighting}
\end{Shaded}

\begin{verbatim}
## [1] 5848.296
\end{verbatim}

\begin{Shaded}
\begin{Highlighting}[]
\FunctionTok{print}\NormalTok{(}\StringTok{\textquotesingle{} \textquotesingle{}}\NormalTok{)}
\end{Highlighting}
\end{Shaded}

\begin{verbatim}
## [1] " "
\end{verbatim}

\begin{Shaded}
\begin{Highlighting}[]
\CommentTok{\# Expected value using the theoretical definition}
\FunctionTok{sum}\NormalTok{(x}\SpecialCharTok{*}\NormalTok{p)}
\end{Highlighting}
\end{Shaded}

\begin{verbatim}
## [1] 675
\end{verbatim}

\begin{Shaded}
\begin{Highlighting}[]
\CommentTok{\# Variance of the earnings using the thoretical definition}
\NormalTok{x }\OtherTok{=} \FunctionTok{c}\NormalTok{(}\DecValTok{0}\NormalTok{, }\DecValTok{500}\NormalTok{, }\DecValTok{1000}\NormalTok{, }\DecValTok{1500}\NormalTok{, }\DecValTok{2000}\NormalTok{)}
\NormalTok{p }\OtherTok{=} \FunctionTok{c}\NormalTok{(}\FloatTok{0.28}\NormalTok{, }\FloatTok{0.27}\NormalTok{, }\FloatTok{0.315}\NormalTok{, }\FloatTok{0.09}\NormalTok{, }\FloatTok{0.045}\NormalTok{)}
\FunctionTok{sum}\NormalTok{((x }\SpecialCharTok{{-}} \FunctionTok{sum}\NormalTok{(x}\SpecialCharTok{*}\NormalTok{p))}\SpecialCharTok{\^{}}\DecValTok{2} \SpecialCharTok{*}\NormalTok{ p)}
\end{Highlighting}
\end{Shaded}

\begin{verbatim}
## [1] 309375
\end{verbatim}

\begin{Shaded}
\begin{Highlighting}[]
\CommentTok{\# Standard deviation of the earnings using the theoretical definition}
\FunctionTok{sqrt}\NormalTok{(}\FunctionTok{sum}\NormalTok{((x }\SpecialCharTok{{-}} \FunctionTok{sum}\NormalTok{(x}\SpecialCharTok{*}\NormalTok{p))}\SpecialCharTok{\^{}}\DecValTok{2} \SpecialCharTok{*}\NormalTok{ p))}
\end{Highlighting}
\end{Shaded}

\begin{verbatim}
## [1] 556.2149
\end{verbatim}

\begin{Shaded}
\begin{Highlighting}[]
\NormalTok{x }\OtherTok{=} \FunctionTok{c}\NormalTok{(}\DecValTok{0}\NormalTok{, }\DecValTok{500}\NormalTok{, }\DecValTok{1000}\NormalTok{, }\DecValTok{1500}\NormalTok{, }\DecValTok{2000}\NormalTok{)}
\NormalTok{p }\OtherTok{=} \FunctionTok{c}\NormalTok{(}\FloatTok{0.28}\NormalTok{, }\FloatTok{0.27}\NormalTok{, }\FloatTok{0.315}\NormalTok{, }\FloatTok{0.09}\NormalTok{, }\FloatTok{0.045}\NormalTok{)}

\CommentTok{\# Deviations from the expected value}
\NormalTok{x }\SpecialCharTok{{-}} \FunctionTok{sum}\NormalTok{(x}\SpecialCharTok{*}\NormalTok{p)}
\end{Highlighting}
\end{Shaded}

\begin{verbatim}
## [1] -675 -175  325  825 1325
\end{verbatim}

\begin{Shaded}
\begin{Highlighting}[]
\CommentTok{\# Squared deviations from the expected value}
\NormalTok{(x }\SpecialCharTok{{-}} \FunctionTok{sum}\NormalTok{(x}\SpecialCharTok{*}\NormalTok{p))}\SpecialCharTok{\^{}}\DecValTok{2}
\end{Highlighting}
\end{Shaded}

\begin{verbatim}
## [1]  455625   30625  105625  680625 1755625
\end{verbatim}

\begin{Shaded}
\begin{Highlighting}[]
\CommentTok{\# Expected value of the squared deviations from the expected value}
\FunctionTok{sum}\NormalTok{((x }\SpecialCharTok{{-}} \FunctionTok{sum}\NormalTok{(x}\SpecialCharTok{*}\NormalTok{p))}\SpecialCharTok{\^{}}\DecValTok{2} \SpecialCharTok{*}\NormalTok{ p)}
\end{Highlighting}
\end{Shaded}

\begin{verbatim}
## [1] 309375
\end{verbatim}

\end{document}
